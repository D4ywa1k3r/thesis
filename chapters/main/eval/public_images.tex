\section{Public available images}
\label{ch:eval:public_images}
Public images are stored and available on container registries. Container registries are provided by several cloud providers like Docker Inc.(DockerHub), Google(GCR), Amazon (ECR) and others as well.   
Each cloud provider offers a private area for users/groups to make the images accessible only to them. However the provided prototype only supports the query of public container registries without authentication.
DockerHub provides a large bundle of images provided by the community and is the standard container registry for public images. A simple strategy is used to find potential candidates from DockerHub to scan these images.
The search on DockerHub includes mainly backend technologies that expects a communication to remote endpoints. This in turn requires secure communication using by mechanisms like RSA for example.	
An additional filter is set to fetch only non official DockerHub images. These images are not proved officially by DockerHub. Third-party images that are not subject to verification theoretically have a high potential of vulnerabilities.
The search is performed manually without programmatically API queries. However the scan is autonomous, as the prototype only needs the name of the suspicious image.
The scan is performed one by one. One of them is a commonly used candidate with more than 10 million downloads. This images was updated 5 month ago, in Oktober 2019. 
\graffito{Note: The SHA is calculated over every image-layer by a special algorithm. This defines an image explicitly and makes clear which image was used exactly.}
The image is called \textit{nodered/node-red-docker} and has the following SHA \textit{sha256:0bd9a1d2200474e7471bada2eb633f7193320ee47cb3b8aa34326d19f7f485c6}.
Finally the analysing output can be seen in the following listing.
\lstinputlisting[caption={Result RSA keys analyse}, captionpos=b, label={ch:eval:public_images:result_public}]{chapters/main/eval/listings/result_public.txt}
8 private keys have been found in the Docker image. For each RSA key found, the corresponding folder was displayed. The folder structure indicates that these keys are mostly used by the MQTT protocol.
The MQTT folder hierarchy displays RSA test keys as well as potential productive client RSA keys. A further investigation can help to associate this private key.
The other RSA keys would also be possible candidates that pursue a certain protection goal. As an example the tls-key.pem may be sensitive since a TLS private key can be used to generate certificate signing requests (CSR), and later to secure and verify connections using the certificate created per that request. All these private keys needs a further investigation in order to obtain possible sensitive informations.

This section showed that the prototype can also be applied to public images. This confirms a possible universal application, since locally created images are not always true.
