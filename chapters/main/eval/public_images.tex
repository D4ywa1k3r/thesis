\section{Public available images}
\label{ch:eval:public_images}
Public images are stored and available in container registries. 
Container registries are provided by several cloud providers.
Examples are the following providers and their solutions.
\begin{itemize}
    \item Docker Inc. - DockerHub
    \item Google - Google Container Registry(GCR)
    \item Amazon - Elastic Container Registry(ECR)
\end{itemize}{}    
There are many other providers and solutions available. 
Each cloud provider normally offers a private area to make images only accessible to special users and groups. 
A valid authentication is necessary to get access to these locked images.
However the provided prototype only supports the query of public container registries without authentication.

DockerHub provides a large bundle of images provided by the community and is the standard container registry for public images.
This platform therefore is a very good candidate for the designed prototype.

A simple strategy is used to find potentially suspicious images from DockerHub.
The search in DockerHub includes mainly backend technologies that expects a communication to remote endpoints.
This in turn requires secure communication using mechanisms such as RSA.
An additional filter is set to fetch only non official DockerHub images.
These images are not officially proved by Docker Inc.
In theory images from non-verified third party vendors have a higher potential of vulnerabilities than official ones.
The search is performed manually without programmatically API queries. 

However the scan is autonomous because the prototype only needs the name of the suspicious image.
The scan is performed one after the other without any parallel process execution.

One of the scans was performed on a frequently used image with more than 10 million downloads.
This image was updated in October 2019. 
\graffito{Note: The SHA is calculated over every image-layer by a special combination of hash calculations. 
This defines an image explicitly and makes clear which image was used exactly.}
The image is called \textit{nodered/node-red-docker} \\and has the following SHA \\\textit{sha256:0bd9a1d2200474e7471bada2eb633f7193320ee47cb3b8aa34326d19f7f485c6}.
\\The console output of the scan can be seen in Listing \ref{ch:eval:public_images:result_public}.
\lstinputlisting[caption={Result RSA keys analyze}, captionpos=b, label={ch:eval:public_images:result_public}]{chapters/main/eval/listings/result_public.txt}
The result shows that 8 private keys have been found in the Docker image.
For each RSA key found the corresponding folder is displayed.
The folder structure indicates that these keys are mostly used by the MQTT protocol.
It can be assumed that the keys found are only for test and demo purposes. This can be deduced from the folder structure.
Further investigation may help to assign this private key. 
It cannot be excluded that these are potentially secret keys.
This also applies to the other private keys.
All these private keys need a further investigation in order to obtain possible sensitive informations.

This section showed that the prototype can also be applied to public images.
This confirms a possible universal application if the prototype shall be developed as a product. 
