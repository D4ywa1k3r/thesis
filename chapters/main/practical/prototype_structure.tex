\section{Prototype structure}
\label{ch:practical_realization:prot_struct}
The following project structure in listing \ref{ch:practical_realization:struc} gives an overview of the prototype how it was built in this paper.
\lstinputlisting[caption={Prototype structure in Python}, captionpos=b, label={ch:practical_realization:struc}]{chapters/main/practical/listings/struc.txt}

The root folder contains the main program the analysing\_manager.py, a modules folder and the venv directory.
The venv directory builds the virtual environment feature of Python, while the modules folder contains the essential modules which represents the core function of the prototype, namely obtain, preprocessing, mount and scan. These python-modules represents the modules in \ref{ch:theory:analysing process} from the theoretical concept. These are managed and used by an analysing\_manager module which lies in parent (root) folder. 

The analysing\_manager module is mainly in interaction with the core modules. The core modules are deliberately created as distinct units. That makes it easier to maintain the prototype. Changing a single core module instead of a large monolith promises more flexibility and counteracts problems. A replacement of core modules is also possible since the interface to the analysing\_manager is untouched.
Through the analysing manager module exists a central endpoint to request the prototype. This is especially useful for a potential microservice. Restpoints have a single place to interact with and to start the analysing. Also other common technologies like RPC's from Apache thrift or Googles protocol buffer can be used to trigger this single analysing endpoint. 

When the analysing manager is triggered with an image name as input, will automatically start its procedure which is shown in the python snippet
\lstset{language=Python}          % Set your language (you can change the language for each code-block optionally)
\begin{lstlisting}[]  % Start your code-block
	
    analysing = AnalManager(img_name)
    analysing.prepare_environment()
    analysing.preprocess()
    if analysing.necessary:
         analysing.mount()
         analysing.examine_deeply()
    
\end{lstlisting}
First it starts with analysing.prepare\_environment() where the obtaining module \ref{ch:practical_realization:implementation:obtain} is doing its job. The Docker image will be downloaded by this module. Afterwards the preprocessing will take place through the call of the preprocess method. The preprocessing will be done by the implemented preprocessing module \ref{ch:practical_realization:implementation:preprocessing}.
Finally the analysing manager can decide if a further investigation is necessary. If so the mount and scan module will be instantiated. The synchronous mount call, as well as the "examine" call is is executed consecutively. The final result of the scan module is just printed out into the stdout stream of Linux.
\graffito{Note: Stdout, also known as standard output, is the default file descriptor where a process can write output.}
When secrets are found, the secret will be printed to the stdout stream. When no secret was found, a standard info message will be piped into the stdout buffer.

This section has given an overview about the structure of the prototype in general. The following section will introduce the implementation of the core modules.

