\section{Evaluation of access methods}
\label{ch:theory:access_eval}
This segment examines the first building block of the core concept of image analysis.
It will evaluate and decide which methods have to be used to get file access to a Docker image in general.

Basically a Docker image is distributed arbitrarily and it cannot be assumed that the Dockerfile can be accessed.
Therefore only the pure availability of the image is realistic.
The first key question is what is a universal method of access to an image in order to perform a depth analysis.
In conclusion the architecture of Docker images leads to three obvious possibilities that are described in the following subsections.
These different approaches have advantages and disadvantages.
One key feature of the access method is a well working interaction with the necessary analyzing module.
Another key factor represents a necessary modification of the image. 
A modification can have a negative effect on the entire scan workflow in general.
Modification basically violates integrity and has to be avoided.
This has to be considered deciding on which access method has to be finally used.

\subsection{Additional image layer} 
\label{ch:theory:access_eval:additional}
The access is made through a new additional layer which contains and also adds a program to the obtained image. 
The program can work on the entire file system when the image is being initiated as a container.
Preprocessing on the original image is mandatory before the new layer is being added. 
The first result has to be temporary saved. 
This includes all the metadata informations that are useful for the scan. 
These metadata are explained more deeply in section \ref{sec:intro:docker_image:docker_img:meta}. 
A change of the base image would lead to a loss of these important informations.

After this processing is done a new additional layer can be added with the analyzing program.
Finally the program would need an endpoint to save the results. 
The result has to be saved on a permanent storage due to the nature of containers.

\subsection{Tarball approach} 
\label{ch:theory:access_eval:tarball}
The idea behind this approach is to pipe a running container into a tarball. The container must be started and remain online until all information are extracted and stored in an archive.
After exporting the container can be deleted immediately because the processing only takes place on the tarball.
This archive contains the complete file system including the writable container layer. The archive can be analyzed afterwards by a program. This program can save the results locally or deliver it to an endpoint.

\subsection{Direct access} 
\label{ch:theory:access_eval:direct_access}
In theory direct access to the image is also possible, since an image is present on the local system before it is started as a container.
The background chapter showed that a Docker image is just a stack of several image layers. 
The direct access to the image as a whole needs a overlay-mount on the host system itself.
Necessary information to mount the overlay correctly has been demonstrated in section \ref{sec:intro:docker_image:docker_img}.
The mandatory lower directories for the overlay have to be examined. 
These can be determined through the locally provided information.
Other mandatory directories of an overlay have to be created accordingly.
These information have to be used in order to create the overlay-mount finally. 
Lastly the program that performs the analysis can access the mount point and browse through the union file system.

\subsection{Decision of access method} 
\label{ch:theory:access_eval:decision_access}
These mentioned approaches have advantages and disadvantages. 	
The access approach in subsection \ref{ch:theory:access_eval:additional} has the drawback of an additional layer and requires a remote communication in order to save the results.
It also needs a somewhat of a copy of the image in order to save the meta-information. 
The modification of the base image leads to a higher complexity and effort. 

The second approach from subsection \ref{ch:theory:access_eval:tarball} only has the image as a base and does not modify anything. 
It only needs a start of a container temporarily. 
Unlike in \ref{ch:theory:access_eval:additional} the container does not have to run during the analyzing process. 
However the fact of starting a container is still a drawback because it can lead to an initially undefined consumption of resources.
The direct access method from subsection \ref{ch:theory:access_eval:direct_access} performs the analysis without starting a container. 
Conclusively no additional container load exists on the host.

Lastly the third approach has the big advantage to access to the image directly through the filesystem. 
This access method is used in this theoretical concept since direct access to the image via the file system requires the least effort from a logical perspective.