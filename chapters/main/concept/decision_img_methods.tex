\section{Evaluation of access methods}
\label{ch:theory:access_eval}
This section examines the first building block of the core concept of image analysis. It will evaluate and decide which methods has to be used to get file access to an image in general.
Basically an image is distributed arbitrarily and it cannot be assumed that the Dockerfile can be accessed.
Therefore, only the pure availability of the image is realistic. The first key question belongs to get an universal access to an image in order to perform a depth analysis.
The architecture of Docker images leads at the end to three obvious possibilities which are described in the following subsections.
These different approaches have advantages and disadvantages. The key feature of the access method is a well working interaction with the necessary analysing module. When the file access method needs big effort or it modifies an image in order to embed necessary actions that much, it can have a bad impact for the whole scanning workflow in general. This has to be considered when the decision is made which access method has to be used finally.

\subsection{Additional image layer} 
\label{ch:theory:access_eval:additional}
The access is made through a new additional layer which contains and adds a program to the obtained image. When the image is initiated and started as a container, the program can work on the entire file system and get the job done.
Before the new layer is added, preprocessing on the original image is mandatory and this first result has to be temporary saved. That includes all the metadata informations which are necessary for the scan. These metadata were explained in \ref{sec:intro:docker_image:docker_img:meta} and these metadata will be used by the processing module later in \ref{ch:practical_realization:implementation:preprocessing}. A change of the base image would lead to a lost of these important informations. When this processing is done, a new additional layer can be added with the analysing program .
Finally, the program would need an endpoint, where the result is saved. The result has to be saved on a permanent storage due to the nature of containers, because the results are removed after stopping the container. 

\subsection{Tarball approach} 
\label{ch:theory:access_eval:tarball}
The idea behind this approach is to pipe a running container into a tarball. The container must be started and remain online until all informations are extracted and stored in an archive.
After export the container can be deleted immediately, because the processing only takes place on the tarball.
This archive contains the complete file system, including the writable container layer. The archive can be analysed afterwards by a program. This program can save the results locally or deliver it to an endpoint.

\subsection{Direct access} 
\label{ch:theory:access_eval:direct_access}
Theoretically, direct access to the image is also possible, since an image is present on the local system before it is started as a container.
The background chapter showed, that a Docker image is just a stack of several image layers. The direct access to the image as whole needs a manual overlay-mount on the host system itself.
Necessary informations to mount the overlay correctly is proved by the the background chapter \ref{sec:intro:docker_image:docker_img}.
For the overlay the mandatory lower directories have to be examined. The sequence of the directories in the chain is also important. These can be examined through the provided informations which are available locally. That includes especially the lowerfile of each Docker layer. This information has to be used in order to get finally the overlay-mount. 
Finally the program that performs the analysis can access the mount point and do its analysis. 

\subsection{Decision of access method} 
\label{ch:theory:access_eval:decision_access}
These mentioned approaches have advantages and disadvantages. 	
In subsection \ref{ch:theory:access_eval:additional} the access approach has the drawback of an additional layer and requires a bi-directional interaction between host and container or container and a specific endpoint in order to save the results. Also it needs kind of copy of the image in order to save the meta-informations. The modifying of the base image leads to a higher complexity and effort. 
The second approach from \ref{ch:theory:access_eval:tarball} has only the image as a base and does not modify anything on the image and only needs a start of a container temporarily. Unlike in \ref{ch:theory:access_eval:additional} the container has not to run during the analysing process. However, the fact of starting a container is still a drawback because it can lead to an initially undefined consumption of resources.
With direct access method from \ref{ch:theory:access_eval:direct_access}, the analysis is performed without starting a container and therefore no additional container load exists on the host.
Finally the third approach has the big advantage to access to the image directly through the filesystem. 
Since direct access to the image via the file system requires the least effort from a logical perspective, this access method is used.