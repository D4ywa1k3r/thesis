\section{Scope of secrets}
\label{ch:theory:scope_secrets}
In order to steer the work in a specific direction it is necessary to set a scope of secrets which has to be discovered. 
A brief definition of this topic will be evaluated in this segment.

Software developers and system engineers are the main stakeholders who create container images regularly. 
Nowadays the technical environments are often cloud services. 
Cloud service players are Amazon, Microsoft and Google as an example. 
The following list points out the central common features of these global players.
\begin{itemize}
\item Compute Engine - Includes virtual machines or clusters
\item Kubernetes Engine - Includes Kubernetes components
\item Storage - Includes several types of databases and hardware provisioning
\item Diverse - Networking, Monitoring, Artificial Intelligence, BigData, etc ...
\end{itemize}

This is a large bundle of functions provided by each individual cloud provider that can be used within a container and thus in an image.
There are many other APIs accessible, apart from the obvious features. For example Google offers more than 100 APIs for developers \cite{gapi}.
Fortunately each cloud provider has Identity and Access Management (IAM) integrated for providing the principle of least privilege. 
But this is at the discretion of the operator.

In order to perform actions via these API's and services, authentication is required beforehand. 
Google, Microsoft and Amazon have established several kinds of authentication. 
The developer has to choose application credentials based on what the application needs and where it runs.
The ranges of credentials is big and contains amongst others credentials API keys, OAuth 2.0 client authentication, environment-provided service accounts and other types of tokens which are derived from an associated technical user.
Since access keys are nowadays often used by developers directly in the code, the API token gets a special consideration in this paper.
It only plays a secondary role, whether the token comes from Google, Microsoft or Amazon. 
It is important to recognize a corresponding architecture or scheme of the token when it comes to the analysis.
In this thesis an API token from Amazon is being investigated. An adaptation to other tokens is also possible when the schema is determined and adapted.
Amazon itself uses a combination of an access key and secret token which is normally directly used in the code.

Most of all solutions can also be used by subscribing to these services directly from the associated vendor.
As a last resort most of all available solutions can be maintained bare-metal.
In every case the authentication depends on possible options of the software itself. 
Simple authentication via user name and password is still common, as well as authentication via certificates.
Certificates themselves are flexible, versatile to use and therefore popular. 
The asymmetric mechanism behind this is usually RSA. 
RSA is widely used and is still the state of art when confidentiality or authenticity is needed. 
The private key is the sensible element in a RSA key pair and is finally responsible for the protection goals.
This key turns into a second popular candidate in this work due to the popularity and important use cases.

RSA keys are usually created with client tools like openssl or ssh-keygen.
The folder and filename can be changed with passing correct command line arguments to the programs which makes the place and name of the private key arbitrary.
The key file can be placed and named wherever the developer sees the necessity. 
The program which requires the key only needs a correct configuration to find the keyfile. 
Only the content of the keys counts and has to be untouched. 
Tools like openssl and ssh-keygen have in common that generated keys are stored on the filesystem in a dedicated file. 
It might be theoretically possible to extract these keys as a plain string and integrate it in a source code, but it is a serious design mistake which needs a great deal of effort. 
Therefore the scope of this work for private keys is only on file system level with consideration of arbitrary locations and names of the fille itself.

It has to be considered that RSA keys should normally contain a passphrase to provide additional security in case someone steals the private key file.
The passphrase is just a key being used to encrypt the file that contains the RSA key using a symmetric cipher (usually DES or 3DES). 
The used symmetric cipher can be viewed by reviewing the header of the private key. 
The file must first be decrypted with the decryption key to use the private key for public key encryption.
This decryption key must also be made available to the container if the private key itself is password protected.
Otherwise a password request in the container occurs and the container does not work automatically.
The passphrase can be accessed from different sources like a file, an environment variable or another streams.
This additional decryption passphrase can be injected during runtime into the container or as a static file itself into the image. 
Those passphrases are arbitrarily chosen and do not follow a syntax.
Therefore no key based search or pattern-based search is possible to find these passwords. 
This is another hurdle to obtain the password. 
However the password is still available if it is statically integrated into the image. 
A runtime integration offers the same hurdle as a dictionary attack to obtain this passphrase.
The additional protection is not considered in this work because of these two constraints.
 
The scope of this theoretical concept is still to find secrets with a fix schema.

This can now be summarized as follows. The API-access token from Amazon is a set of sensitive key pair while the RSA private key alone is very sensible. 
Due to the nature of inserting API tokens in plain text into source code, this is a different level compared to RSA private keys that exists on a file system.
These two types of keys define the target to be detected in the images.
