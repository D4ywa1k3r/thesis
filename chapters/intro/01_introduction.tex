\chapter{Introduction}
\label{ch:intro}

Nowadays the use of application containers has become indispensable. For developers and system engineers the advantages of those application containers are obvious. Container technology provides isolation and portability for any built application. Applications are usually business software instead of consumer software. This includes web-applications, databases and further large systems. 

In enterprise environments several containers often form one application. Therefore the given isolation has to be partially removed in order to enable intercommunication between these containers. Credentials have to be integrated to guarantee a certain degree of security. Credentials such as passwords or other tokens can establish a more or less secure connection to a specific remote endpoint. This finally results in a runnable application stack with secure communication between a container-stack.

This presupposes developers to adhere to security standards when creating a container-stack. A container is usually an instance of an image. In other words the image builds the base of every container. A fully developed image is usually found on online platforms, known as container-registries. These registries are normally available for the public. Anyone who has direct access to the image gets access to the file system within the image automatically. If a developer has incorporated secrets statically into the image to enable secure communication over the Internet, they are easily visible to attackers. These secrets can be tapped and exploited. After all the embedded secrets are obsolete. That is a fatal problem in container images in general.

The trust of images builds an additional problem \cite{to_docker_or_not}.
The image integrity is generally not guaranteed and can be leveraged by the popular man in the middle attack(MITM). There is a signing process necessary with e.g. Docker content trust or a non-central blockchain approach \cite{Xu2018}. This leads to a guaranteed integrity assurance.

Outdated software-packages are also a problem for container images. Vulnerabilities of outdated packages are common and public available on Common Vulnerabilities and Exposures(CVE) systems \cite{10.1007/978-3-319-94289-6_8}. In theory this problem can be solved by hard work through a consequent patching mechanism. 

However this paper focuses only on the problem of embedded secrets which are totally necessary for containers to communicate in a secure manner.

Nowadays Docker is a famous container technology. Docker allows deployments in cloud native environments and on bare metal machines. Possible supported operating systems are for example Windows, MacOS and Linux.
The latest IT news reports that Docker is currently undergoing a radical change. There might be alternative solutions in the future \cite{docker_heise}. This work provides exemplary samples with Docker, since containers are still promising. Since Linux fundamentals form the basis basically, this work can be adapted to other container technologies.

Currently the computer science has developed some approaches to discover secrets in general \cite{7180102}. Related work to detect secrets in container images is not given yet.
That is the basis on which the thesis starts the work with Docker as an example.

%
% Section: Motivation
%
\section{Motivation}
\label{sec:intro:motivation}
Secrets like passwords and other authentication tokens are used by almost every software that needs a secure communication in an unfaithful environment. The main goal of secrets are to tackle challenges like confidentiality, integrity, authenticity and accountability. A key loss would lead to a collapse of the above protection objectives in general. A key-leak is inevitable as soon as a key is integrated into an image.
The attacker only needs to launch the container from the image to gain access to the file. Afterwards a scan of the file system for tokens can be made easily. 

Probably a developer is trained in security standards and will not integrate tokens statically. But developers are also error-prone people and there is no technical prevention for the static integration of secrets. 
The development of a concept to detect embedded secrets in container images allows better control of the secrets used and catches possible errors of the developer. This concept does not work preventively, but can be used as a control instance before further processing or delivery. 
%
% Section: Ziele
%
\section{Goal of this work}
\label{sec:intro:goal}
This work develops an approach to detect embedded secrets in container images. This is tackled by a theoretically developed concept. This concept includes the access method to the container image and an analysis workflow for the detection of defined secrets. The concept is limited to two important secret-types to set a certain scope.
A further goal is the prototypical implementation of the concept. The prototype is also evaluated by applying realistic scenarios.
Finally, clear results should emerge from the prototype. The prototype is available on GitHub and on the attached CD.

%
% Section: Struktur der Arbeit
%
\section{Structure}
\label{sec:intro:structure}
Before falling into details it is worth to have a structure of this paper in mind. This helps to understand the following work in the best possible way. This thesis is basically aimed at everyone who is interested in computer science and has basic knowledge of Linux architecture and container virtualization.

The background chapter starts with container technology introduction in general related to Linux. 
That includes a description to containers in general with a brief comparison to the classical virtualization. Then the Linux namespaces are explained, since they are used by every container. 
For a better understanding of the container concept, the topic of C-groups is also addressed
Afterwards the topic UnionFS is explained, because this knowledge is necessary for understanding the following container images.
The UnionFS and Image sections are essential as they constitute the basic know-how of the work.

The third chapter gives an introduction to key-leak techniques in general.
Scientifically elaborated documents are reviewed to get an overview of possible key-leak techniques.
It is considered which existing methods can be used and adapted for the theoretical concept.

The fourth chapter contains the theoretical concept to detect embedded secrets in images. The concept starts with setting a scope of secrets to detect. Afterwards the image access is defined by a proper comparison.
Finally the analysing workflow is defined based on the defined image access method.
The analysing workflow represents the core work which represents the scientific delta.

The fifth chapter describes the practical realization of the theoretical concept. A prototype is built and described which exactly reflects the functions and modules from the concept. 

Finally, the prototype is evaluated with the help of valid use cases. First self-made images are applied to the prototype and then public images. The results are discussed briefly and concisely in the same chapter.

The thesis is concluded by a conclusion chapter and future-work chapter.	
The next chapter will introduce as mentioned with containerization insights.
