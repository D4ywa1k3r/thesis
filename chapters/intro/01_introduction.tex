\chapter{Introduction}
\label{ch:intro}

The use of application containers has become indispensable today. The advantages of those application containers are obvious to developers and system engineers. Container technology provides isolation and portability for any built application. Applications are consumer software, but most of all software that is needed in the business environment, such as large databases and web applications.

In enterprise environments, several containers also often form one application. This means that the given isolation has to be partially removed and intercommunication these containers has to be enabled in order to provide a running application stack. Therefore credentials have to be integrated to guarantee a certain degree of security. Credentials such as passwords or other tokens can establish a more or less secure connection to a specific remote endpoint. 

This now presupposes that developers are security-conscious and adhere to security standards. A container is usually an instance of an image and the image builds the base of every container. A fully developed image is usually found on online platforms, known as container-registries. These registries are normally available for the public. Anyone who has direct access to the image also automatically gets access to the file system within the image. If a developer has integrated secrets statically in order to create a secure communication across the internet into the image, these are easily visible to curious people or attackers. These secrets can be tapped by attackers, exploited and thus become obsolete. That is a fatal problem in container images in general.

In addition to this problem, there are others such as the trust of images regarding integrity \cite{to_docker_or_not}.
A popular attack is a man in the middle attack in this case. There is a signing process necessary with e.g. Docker content trust or a non-central blockchain approach \cite{Xu2018}. 

Even obsolete packages are a problem for container images. Obsolete packages often contain vulnerabilities that are published and visible on Common Vulnerabilities and Exposures(CVE) systems \cite{10.1007/978-3-319-94289-6_8}. In theoretical, this problem can be solved by hard work through a consequent patching mechanism. 

However, this paper will only focus on the problem of the embedded secrets in container images, which are totally necessary for containers to communicate in a secure manner with other containers or services.

Nowadays Docker is a famous container technology. Docker allows deployments in cloud native environments, on bare metal machines on Windows Systems, Mac OS and Linux operating systems. 
Latest IT-news proves that Docker is currently undergoing a radical change and that there will be alternative solutions in the future \cite{docker_heise}. Since containers are still promising for the future, this work will provide exemplary samples with Docker.

However, this work can be adapted to other container technologies, since Linux core features still form the basis.

Currently, the computer science has developed some approaches to discover secrets in general \cite{7180102}. Related work to detect secrets in container images is not given yet.
That is the basis on which the thesis starts its work with Docker as an example.

%
% Section: Motivation
%
\section{Motivation}
\label{sec:intro:motivation}
Secrets are used by almost every software that needs a secure communication in an unfaithful environment. For example, software that transfers sensitive data through the internet to a service-endpoint. The main goal of secrets are to tackle challenges like confidentiality, integrity, authenticity and accountability. A key leak would lead to a collapse of the above protection objectives in general.
If a key has been integrated into a container image, a key leak is inevitable.
It is not difficult to steal keys from an image, since the attacker only needs to launch the container from the image to gain access to the file. The attacker can afterwards scan the file system for tokens, or check the source code or encapsulated archives and binaries for secrets. 

Even if it is unknown and not allowed to the developer, nothing and no one can prevent technically the static integration of secrets. It is a pure failure of the developer.

The development of an image analysis for embedded secrets allows a better control of the secrets used. This does not work preventively but can be used as a control instance before further processing or delivery. 
%
% Section: Ziele
%
\section{Goal of this work}
\label{sec:intro:goal}
This work tries to develop an approach to detect embedded secrets in container images. This is approached structurally by a theoretically developed concept, which contains various elementary building blocks. These include access to the container image as well as analysis and other points. The concept limits the goals to be aimed at to two important secrets in order to set a certain scope.
In addition, new evaluated theoretical concepts are presented in this article using prototype applications. The prototypes are available on GitHub and on the attached CD.
Finally, this prototype is applied in practice to obtain the result of the theoretical concept.
%
% Section: Struktur der Arbeit
%
\section{Structure}
\label{sec:intro:structure}
Before falling into details its good to have a structure of this paper in mind. This helps for the following work in the best possible way.
This work is basically targeted for everyone who is interested in computer science with basic knowledge about Linux architecture and container virtualization.

In this second chapter right after this section, it starts with container technology introduction in general related to Linux. The difference to classical virtualization is described briefly. Linux namespaces are also explained, because they represent an important functionality in the container area followed by a brief introduction to cgroups.
There follows a deep dive into container images which form the basis for the work. This is an essential section that contains the functionality + usage of the union mount file system OVERLAY2.

The third chapter introduces related work regarding key leak techniques in general with source codes and file systems. These techniques will be adapted in the following chapters.

The fourth chapter will contain a theoretical approaches to analyse the image. That includes the secret types to detect, the image access which has to be used and the processing of the image itself.
This chapter afterwards focuses in particular on the practical development of evaluated methods to verify and compare the elaborated approaches.

Finally, there will be a summary of the collected results and a solid basis for future scientific work.

In the next chapter it will go on as mentioned with containerization insights.
