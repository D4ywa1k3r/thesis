\section{Containerization paradigm}
\label{sec:intro:containerization}
%Definition Container
% Short overview of how a container works
% Short architecture
% properties
%
% 
In theory the paradigm of containers is very simple. 
Containers are basically operated as stateless and separate units. 
These units can fulfill certain tasks by acting autonomously as isolated software components.
The paradigm is integrated by certain kernel functionalities. 
These functionalities with the jails concept in BSD have been introduced in the early 2000s \cite{Souppaya:2017aa}.
In 2008 more functionalities have been popularized in a user-friendly way in the Linux kernel.
The framework is called Linux Containers(LXC's) \cite{Souppaya:2017aa}.
However the usage of LXC's can become quite complex.
The using of LXC has become much easier since Docker was introduced as a wrapper.
Docker offers a user-friendly interaction with the LXC framework and has established itself as a state-of-the-art construct in terms of containerization.
Nowadays developers are able to provide containers due to Docker at a much faster pace than with pure LXC's. 
	
Native Linux features are responsible for the encapsulation between host and deployed containers.

Since the introduction of container technology under BSD, native functions form the basic framework of containers.
These necessary isolation and permission concepts are described in the following.