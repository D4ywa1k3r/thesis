\section{Classification and placement}
\label{sec:intro:virt_and_cont}
This section classifies full virtualization and containerization concept shortly by demonstrating the architectural difference. 

Containerization is widely used in cloud environments. Containers are almost the foundation of every cloud environment.
However classical virtualization existed long before container concepts. Both concepts have advantages and disadvantages. That's why they are also combined as hybrid concepts.
Figure \ref{fig:intro:diff_container_vm} shows the architectural difference between container technology and full virtualization. 
\begin{figure}[htbp]
 \centering
 \includegraphics[width=0.7\textwidth]{gfx/examples/os_virt_diff}
 \caption{Difference between container and full virtualization}
 \label{fig:intro:diff_container_vm}
\end{figure}
Full virtualization allows it to run an entire guest operating system in a virtual machine (VM) on a host operating system. This is possible through an installed piece of software called hypervisor which is build classically on top of the originally initiated operating system. This virtualization model provides solid security through this additional isolation layer. One obvious drawback is the heavyweight and therefore high resource usage characteristic.
In contrast to classical virtualization a container does not need an additional operating system layer as seen in Figure \ref{fig:intro:diff_container_vm}. The container is just using and sharing the underlying kernel from the host. 
Containerization technology is therefore closer to the underlying host operating system than the classical virtualization strategy. That makes containers more lightweight and therefore flexible. Comparisons between container concepts and classical virtualization with regard to the application purposes are described in \cite{}
The next section gives a closer look at containerization paradigm itself.

