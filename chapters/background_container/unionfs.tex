\section{Union file system}
\label{sec:intro:docker_image:unionfs}
This paper uses Docker as container technology as already mentioned in chapter \ref{ch:intro}.
Having a certain level of knowledge on union file systems is necessary before delving into container and especially Docker image insights.
Union file systems build the basis for container images in general.
This section explains one important file system - the Overlay2 file system.

A union file system represents a file system by grouping directories and files.
There are several union file system available like BTRF, AUFS, ZFS and Overlay2 which are compared in detail in \cite{Tarasov2019}.
Only Overlay2 will be considered in this work because Overlay2 is directly implemented in the Linux kernel \cite{Tarasov2019} and is meanwhile the standard in Docker related to Docker Inc. \cite{docker_storage_driver}.
Basically Overlay2 needs at least 4 directory types to work correctly:
\begin{itemize}
\item Lower directory - usually read-only and can be an overlay itself
\item Upper directory - is normally writable
\item Merged directory - represents the union view of the lower and upper directory
\item Work directory - used to prepare files before they are switched to the overlay/merged destination in an atomic action 
\end{itemize}
The elements of an Overlay2 file system are now described with the help of an example. 
A following folder structure in Listing \ref{intro:overlay:hierarchy} is assumed.
\lstinputlisting[caption={General structure of an Overlay2 file system}, captionpos=b, label={intro:overlay:hierarchy}]{chapters/intro/listings/tree1.txt}
The folder structure contains all the mandatory Overlay2 elements. 
The mount point can be created without additional software packages because Overlay2 is natively supported under Linux since Kernel version 4.52.2 \cite{}.
This is illustrated by the following mount command.
\begin{lstlisting}[label={intro:overlay:mountcmd}]
	mount -t overlay -o lowerdir=./lower1:./lower2,upperdir=./upper,workdir=./work overlay ./merged
\end{lstlisting}
First the command must know what type of file system to mount.
This information is provided by the \textit{-t} switch. In this case it is an overlay type. 
The next flag \textit{-o} allows to add options to mount the specific filesystem.
Each option with associated values is separated by a comma.
The option lowerdir is set to a chain of folders separated by a colon. 
The lowerdir argument takes only the \textit{lower1} and \textit{lower2} directory. 
Then upperdir is set to the \textit{upper} folder of the provided hierarchy. 
The worker option represents a single folder and is set accordingly to the \textit{worker} folder. 
Lastly overlay option needs an argument to provide the union mount on the file system. 
This union view is presented through the \textit{merged} directory.

The behavior of the Overlay2 file system is demonstrated in the following.
The file system as a whole is populated with files as seen in Figure \ref{sec:intro:docker_image:treefilled}.
\lstinputlisting[caption={Overlay2 file structure populated with data}, captionpos=b, label={sec:intro:docker_image:treefilled}]{chapters/intro/listings/tree2.txt}
A file creation in one of the \textit{lower} and \textit{upper} directories is visible as expected in the \textit{merged} directory.
A file or directory object in the \textit{upper} directory tree appears in the overlay. The same object is not visible in the \textit{lower} directory.
Each directory content is merged to create a combined directory object in the overlay.
A file or directory that originates from the overlay is removed from the overlay when it has been removed from the \textit{upper} directory. 
A file or directory that originates from the \textit{lower} directory remains in the \textit{lower} directory when it was removed from the overlay-directory.
In this case a whiteout mark is created in the upper directory. A whiteout takes the form of a character device with device number 0/0 and a name identical to the removed object. 
A whiteout ensures that the object in the lower directory is simply ignored. 
Also a whiteout is not visible in the overlay directory. 

Another important fact about union file system is the use of copy on write strategy \cite{cow}.
A storage driver manager takes care of copying files to the \textit{upper} layer when a file from an underlying layer has been modified. A duplicate is created and the modification takes place. This is an efficient resource management technique because operations may just need a copy instead of creating a new file.
	
The general knowledge about the Overlay2 file system is provided. This is important to understand the key component - the Docker image. This is described in the following.