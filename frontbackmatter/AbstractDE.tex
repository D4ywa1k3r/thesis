%*******************************************************
% Abstract in German
%*******************************************************
\begin{otherlanguage}{ngerman}
\pdfbookmark[1]{Zusammenfassung}{Zusammenfassung}
\chapter*{Zusammenfassung}
Heutzutage wird cloud native computing immer beliebter. Insbesondere die Anwendungscontainer von Docker sind ein wesentlicher Bestandteil des nativen cloud computing. Jeder Container basiert auf statischen Datenstrukturen - sogenannten Abbildern. Diese Abbilder formen ein wichtiges, aber auch kritisches Element in diesem Ökosystem.
Abbilder werden in cloud Umgebungen oder temporär in anderen öffentlichen oder privaten Container-Plattformen gespeichert. Dies führt zu Problemen sobald der Entwickler symmetrische oder asymmetrische Geheimnisse wie SSH-Schlüssel, Klartext-Passwörter, Zertifikate und andere in das Abbild integriert hat.

Die Idee hinter dieser Arbeit ist die Entwicklung eines Ansatzes zur Erkennung privater RSA-Schlüssel und Amazon-Zugriffstokens. 
Beide Token sind allgegenwärtig und werden von Entwicklern in Container Abbildern verwendet. 
Verwandte Arbeiten existieren, um Geheimnisse im Quellcode im Allgemeinen, aber nicht direkt in Container-Bildern zu entdecken. 
Diese Arbeit entwickelt und verfolgt einen theoretischen Ansatz mit prototypischer Umsetzung um zu beweisen, dass eine Anwendung bekannter Key-Leak-Techniken zur Erkennung eingebetteter Geheimnisse in Docker Abbildern möglich ist.
\end{otherlanguage}
